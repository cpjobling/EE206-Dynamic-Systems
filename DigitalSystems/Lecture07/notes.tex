%% Digital Systems
%% The z transform
\def\FileDate{98/11/04}
\def\FileVersion{1.0}
% ----------------------------------------------------------------
% Notes pages *********************************************************
% ----------------------------------------------------------------

\begin{slide}\label{slide:l8s1}
\heading{Definition of the $z$ Transformation} 
\begin{itemize}
 \item The $z$ transformation has the same role in digital systems
 that the Laplace transform has in continuous systems.
 \item $z$ transformation of a digital signal $v$ is defined as \[V =
 V(z) = \mathcal{Z} v = \mathcal{Z} \{ v_{k} \}=\sum_{k=0}^{\infty}
 v_k z^{-k}.\] 

 \item Similarly, $v$ is the \emph{inverse $z$
 transform} of $V$, or
   \[\mathcal{Z}^{-1} V.\]

\end{itemize}

\end{slide}


\begin{slide}\label{slide:l8s1a}
\heading{$z$ Transform is a Summed Series (1)} 

\begin{itemize}

\item Obtaining $z$ transforms involves summing series. Often this is a
\emph{binomial} series \[(1+x)^r = 1 + rx + \frac{r(r-1)}{2}x^2 +
\cdots + \frac{r(r-1)\cdots(r-n+1)}{n!}x^n + \cdots\]

\item The transform pairs for common functions
are put into tables so that it is not necessary to sum a series in
most cases.

\end{itemize}

\end{slide}

\begin{slide}\label{slide:l8s1a1}
\heading{Example 1: $z$ Transform of The Unit Step} 

\begin{itemize}

\item If $v$ is the digital sequence $\{\epsilon_k\}$, that is
\[1\ 1\ 1\ 1\ 1\ \ldots\]

\item $z$ transform is
\[\sum_{k=0}^{\infty} 1\times z^{-k} 1x= 1 + z^{-1} + z^{-2}+\cdots\] 

\item this is summed
as a binomial series as \[(1-z^{-1})^{-1} = \frac{1}{1-z^{-1}} =
\frac{z}{z-1}.\] 

\end{itemize}
\end{slide}

\begin{slide}\label{slide:l8s1b}
\heading{Example 2: $z$ Transform of a Power Series} 

\begin{itemize}

\item Another example which is commonly found is the digital signal $v$
given by $\{v_k\} = \{\alpha^k\}$. 

\item The $z$ transform is
\[\sum_{k=0}^{\infty} \alpha^{k} z^{-k} = 1 + \alpha z^{-1} + \alpha^2
z^{-2}+\cdots\] 

\item $\ldots$ which is summed as a binomial series as
\[(1-\alpha z^{-1})^{-1} = \frac{1}{1-\alpha
  z^{-1}} = \frac{z}{z-\alpha}.\]

\end{itemize}

\end{slide}

\begin{slide}\label{slide:l8s1c}
\heading{Example 3: $z$ Transform of a Continous Signal} 

\begin{itemize}

\item If the digital signal $v$ is generated by sampling the continuous
signal $v(t)$, then the transform $V$ also has a correspondence
with $v(t)$. 

\item For example, if \[v(t) = e^{-at}\] then \[v_k = v(kT)
= e^{-akT} = (e^{-aT})^k.\] 

\item $\ldots$ So, from the previous result, with
$\alpha=e^{-aT}$, we have the $z$ transform as
\[\frac{1}{1-e^{-aT}z^{-1}}=\frac{z}{z-e^{-aT}}.\]

\end{itemize}

\end{slide}

\section*{$z$ Transforms of the Shift Operators}

\begin{slide}\label{slide:l8s2}
\heading{\emph{z} Transform for Forward Shift}

\begin{itemize}
\item  This is similar to the derivative property of the Laplace
  transform.
 \[\mathcal{Z}\triangle v = z V(z) - zv_0\] It also introduces the initial member $v_0$.

\item  We can also show that
  \[\mathcal{Z}\triangle^2 v = z^2 V(z) - z^2 v_0 - z v_1\]
\item and in general
  \[\mathcal{Z}\triangle^r v = z^r V(z) - \sum_{i=0}^{r-1} v_i z^{r-i}.\]

\end{itemize}

\end{slide}

To prove the relationship illustrated in \sref{slide:l8s2} we note
that
\[\triangle v = v_1\ v_2\ v_3\ \ldots\ v_{k+1}\ldots\]
so that
\begin{eqnarray*}
  \mathcal{Z}\triangle v & = & v_1 z^0 + v_2 z^{-1} + v_3 z^{-3} + \cdots
  + v_{k+1}z^{-k}+\cdots \\
& = & z(v_0 + v_1 z^{-1} + v_2 z^{-2} + \cdots
  + v_{k+1}z^-{k+1}+\cdots)-zv_0\\
&=& zV(z) - zv_0.
\end{eqnarray*}
Similar arguments may be used to prove the other relationships
shown in \sref{slide:l8s2}.

\begin{slide}\label{slide:l8s3}
 \heading{$z$ Transform for Backward Shift}
\begin{itemize}
\item  This is similar to the integral property of the Laplace
   transform.
  \[\mathcal{Z}\nabla v = z^{-1} V(z)\]
\end{itemize}

\end{slide}

\section*{The Inverse z Transformation}
\begin{slide}\label{slide:l8s4}
   \heading{Inverse $z$ Transform}
   \begin{itemize}
   \item The inverse $z$ transform of $V$ is $\mathcal{Z}^{-1} V = v$,
     where $v$ is a digital signal, that is a sequence.
   \item This contrasts with the inverse Laplace transformation, which
     gives functions of time.
   \item As the definition of the transform involves $z^{-1}$, through
     \[V=\sum_{k=0}^{\infty} v_k z^{-k},\] it is often useful to
     commence the inversion if $V(z)$ by expressing it as a function of
     $z^{-1}$ rather than of $z$.
   \end{itemize}
\end{slide}

\begin{slide}\label{slide:l8s5}
   \heading{Methods}
   \begin{itemize}
   \item There are several methods for obtaining inverse $z$
     transforms.
   \item They generally involve expressing $V$ as a series involving
     powers of $z^{-1}$, from which the coefficients give the sequence
     $\{v_k\}$ that is $v$.
   \item Alternatively, a partial fraction expansion is used to obtain
     terms from which the inverse transforms may be looked-up from tables.
   \end{itemize}
\end{slide}

The methods will be illustrated for \[V=\frac{4z^3 - 16z}{z^3 - z^2 -
  0.25 z + 0.25}.\]

\subsection*{Direct Method (polynomial division)}
This involves expressing $V$ as a function of $z^{-1}$, and
dividing the numerator polynomial by the denominator polynomial.

First express $V$ as a function of $z^{-1}$ by multiplying numerator
and denominator by $z^{-3}$.
\[ V = \frac{4-16z^{-2}}{1-z^{-1}-0.25 z^{-2} + 0.25 z^{-3}} \]
Then divide \[
\begin{array}{rrrrrrr}
   & +4 & +4z^{-1} & -11z^{-2} & -11z^{-3} & +\cdots  \mathrm{etc}& \\ \cline{2-6}
  1-z^{-1}-0.25
z^{-2} + 0.25 z^{-3} &  \vline\ +4 & . & -16z^{-2} &  & &
\\
   & +4 & -4z^{-1} & -z^{-2} & +z^{-3} &  &  \\ \cline{2-5}
   &  &   +4z^{-1} & -15z^{-2} & -z^{-3} &  &  \\
   &  &   +4z^{-1} & - 4z^{-2} & -z^{-3} & +z^{-4} &  \\
   \cline{3-6}
   &  &           & -11z^{-2} & .         &    -z^{-4} &  \\
   &  &           & -11z^{-2} & +11z^{-3} & +2.75z^{-4} & -2.75z^{-5}
   \\\cline{4-7}
  &   &           &           & -11z^{-3} & -3.75z^{-4} &
  +2.75z^{-5} \\
    &   &           &           & \cdots & \cdots &
  \cdots \\
\end{array}
\]

So \[ V = 4 + 4z^{-1} -11 z^{-2} -11z^{-3} + \cdots\
\mathrm{etc}\] \[ v = \begin{array}{ccccccc}
  4 & 4 & -11 & -11 & \cdots & \cdots & \mathrm{etc}.
\end{array} \]

\subsection*{Indirect Method (partial fraction expansion)}
This also involves expressing $V$ as a function of $z^{-1}$, but
now the denominator is factorized and a partial fraction is
obtained.

So \begin{eqnarray*}
 V & =&  \frac{4-16z^{-2}}{1-z^{-1}-0.25 z^{-2} +
0.25 z^{-3}}\\ &=& \frac{16-64z^{-2}}{4-4z^{-1}-z^{-2} + z^{-3}}\\
&=&
\frac{16(1-2z^{-1})(1+2z^{-1})}{(2-z^{-1})(2+z^{-1})(1-z^{-1})}\\
&=& \frac{\frac{16\times -3\times 5}{4\times -1}}{2-z^{-1}} +
\frac{\frac{16\times 5\times -3}{4\times 3}}{2+z^{-1}} +
\frac{\frac{16\times -1 \times 3}{1\times 3}}{1-z^{-1}}
\\ &=&\frac{60}{2-z^{-1}} -
\frac{20}{2+z^{-1}} - \frac{16}{1-z^{-1}}
\\&=&\frac{30}{1-{1/2}z^{-1}} -
\frac{10}{1+{1/2}z^{-1}} - \frac{16}{1-z^{-1}}
\\
&=& 30\frac{z}{z-{1/2}} - 10\frac{z}{z+{1/2}} - 16\frac{z}{z-1}.
\end{eqnarray*}
Inverse transforming each term from tables gives
\begin{eqnarray*}v&=&30\{(1/2)^k\}-10\{(-1/2)^k\} -16 \{\epsilon_k\}\\
&=& \begin{array}{ccccccc}
  4 & 4 & -11 & -11 & \cdots & \cdots & -16
\end{array}.
\end{eqnarray*}

The direct method of polynomial division is useful for obtaining
the first few members of a sequence, and the indirect method of
partial fraction expansion is useful for obtaining a closed-form
representation of the sequence (if one exists).

Two further properties of the $z$ transform (illustrated in
\sref{slide:l8s10} are useful for finding the initial and final
values of a sequence.
\begin{slide}\label{slide:l8s10}
  \heading{Initial and Final Values of a Sequence}
  \begin{itemize}
  \item \emph{Initial Value Property} \[v_0 = \lim_{z\rightarrow
  \infty}V(z).\]
  \item \emph{Final Value Property} \[\lim_{k\rightarrow \infty}v_k = \lim_{z\rightarrow
      1}\frac{z-1}{z}V(z).\]
  \end{itemize}
\end{slide}

In the example, the initial value is \[\lim_{z\rightarrow
  \infty}\frac{4z^3-16z}{z^3-z^2-0.25z+0.25} = 4.\] The final value is
\begin{eqnarray*}
  \lim_{k\rightarrow \infty}v_k &=& \lim_{z\rightarrow
      1}\frac{z-1}{z}\ \frac{4z^3-16z}{z^3-z^2-0.25z+0.25}\\
 &=& \lim_{z\rightarrow
      1}\frac{z-1}{z}\ \frac{(4)(z)(z-2)(z+2)}{(z-1)(z-0.5)(z+0.5)}\\
&=& \frac{4\times -1 \times 3}{0.5\times 1.5} = -16.
\end{eqnarray*}
%----------------------------------------------------------------
% The end of notes
% ----------------------------------------------------------------
\endinput

% Local Variables:
% TeX-master: "lecture8"
% End:
