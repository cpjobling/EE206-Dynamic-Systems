%% Digital Systems
%% Digital System Response
\def\FileDate{98/11/25}
\def\FileVersion{1.0}
% ----------------------------------------------------------------
% Notes pages *********************************************************
% ----------------------------------------------------------------

As in the case of a continous system, the response of a digital
signal comprises the sum of a free response and a forced response.
The free response is dependent on the initial conditions of a
digital system states, and as these are taken as zero here the
free response is alos zero and will not be considered further. The
response of a digital system with transfer function $H(z)$ to a
digital input signal $u$ is therefore the digital output signal
$y$ given in transform form as
\begin{equation}\label{eq:l10e1}
  Y(z) =H(z) U(z)
\end{equation}
Taking the inverse transform gives the digital system response as
\begin{equation}\label{eq:l10e2}
  y_k = \mathcal{Z}^{-1}Y(z) = \mathcal{Z}^{-1} \left\{H(z)
  U(z)\right\}
\end{equation}
The inverse transform needed to determine the digital system
response is obtained using the inverse z transform methods, e.g.
polynomial division and partial fraction expansion, discussed in a
previous lecture.

\section*{Response to Singularity Signals}

The elemental singularity signals in a digital system response
include the digital impulse signal and the digital step input.

\subsection*{Impulse response}
The digital impulse signal is given by
\begin{equation}\label{eq:l10e3}
  v = \delta = \{\delta_k\}
\end{equation}
where $\delta_0 = 1$ when $k = 0$, and $\delta_k = 0$ otherwise.
Therefore the sequence for the impulse is simply
\[\delta_k = 1,\ 0,\ 0,\ 0,\ \ldots.\]

The transform of the digital impulse signal is
\begin{equation}\label{eq:l10e4}
  V = \Delta = \sum_{k=0}^{\infty}\delta_k z^{-k} = 1
\end{equation}

Consider the system \[ H(z) = \frac{4z^2 - 16}{z^2 - 0.25}\], the
impulse response will be
\[ Y(z) = H(z)\times 1 = \frac{4z^2 - 16}{z^2 - 0.25}\]
We shall determine this response using the partial fraction
expansion.
\begin{eqnarray*}
Y(z) &=& \frac{4 - 16z^{-2}}{1 - 0.25 z^{-2}}\\
     &=& \frac{4(4 - 16z^{-2})}{4 - z^{-2}}\\
     &=& \frac{4(2 - 4z^{-1})(2 + 4z^{-1})}{(2 - z^{-1})(2 + z^{-1})}
\end{eqnarray*}
Assuming a partial fraction expansion of the form \[Y(z) =
\frac{A}{2 - z^{-1}} + \frac{B}{2 + z^{-1}} + C \]  we have
\begin{eqnarray*}
 \frac{4(2 - 4z^{-1})(2 + 4z^{-1})}{(2 - z^{-1})(2 + z^{-1})}
     &=& \frac{A(2 + z^{-1}) + B(2 - z^{-1}) + C(2 - z^{-1})(2 + z^{-1})}{(2 - z^{-1})(2 + z^{-1})}\\
     16 - 64z^{-2} &=& 2A + Az^{-1} + 2B - Bz^{-1} + 4C - Cz^{-2}
 \end{eqnarray*}
Gathering terms and equating coefficients
\begin{eqnarray}
16 &=& 2A +2B + 4C\\ 0 &=& A - B\\ -64 &=& -C
\end{eqnarray}
Hence
\begin{eqnarray}
C &=& 64\\ A &=& B\\ 16 &=& 4A + 256\\ A &=& B = -60
\end{eqnarray}
Thus
\begin{eqnarray*}
    Y(z) &=& 64 -\frac{60}{2-z^{-1}}-\frac{60}{2+z^{-1}}\\
    &=& 64 -\frac{30}{1-1/2 z^{-1}}-\frac{30}{1+1/2 z^{-1}}\\
     y_k& =& \left\{64\delta_k - 30\left(\frac{1}{2}\right)^k - 30
\left(-\frac{1}{2}\right)^k\right\}\\
 &=& \left\{4,\ 0,\ -15,\ 0,\ -3.75,\ 0,\ -0.9375,\ \ldots
 \right\}
\end{eqnarray*}

\subsection*{Step response}
The digital step signal is
\begin{equation}\label{eq:l10e5}
  v = \epsilon = \{\epsilon_k\}
\end{equation}
where $\epsilon_k = 1$ when $k \ge 0$, and $\epsilon_k = 0$
otherwise. Therefore the sequence for the step is simply
\[\epsilon_k = 1,\ 1,\ 1,\ 1,\ \ldots.\]

The transform of the digital step signal is
\begin{eqnarray}\label{eq:l10e6}
  V = E &=& \sum_{k=0}^{\infty}\epsilon_k z^{-k}\\
        &=& \sum_{k=0}^{\infty} z^{-k}\nonumber \\
        &=& 1 + z^{-1} + z^{-2} + z^{-3} + \cdots z^{-n} + \cdots \nonumber\\
        &=& \frac{1}{1-z^{-1}} = \frac{z}{z-1}.
\end{eqnarray}
The step response of the example system is
\[ Y(z) = H(z)\times \frac{z}{z-1} = \frac{z(4z^2 - 16)}{(z-1)(z^2 - 0.25)}\]
We shall determine this response using the partial fraction
expansion.
\begin{eqnarray*}
Y(z) &=& \frac{4z^3 - 16z}{z^3 - z^2 - 0.25z + 0.25}\\
 &=& \frac{4 - 16z^{-2}}{1 - z^{-1} - 0.25z^{-2} + 0.25z^{-3}}\\
\end{eqnarray*}
In the last lecture we showed that the result of the partial
fraction expansion was
\[\frac{30}{1-1/2 z^{-1}} - \frac{10}{1+1/2z^{-1}} -
\frac{16}{1-z^{-1}}\] and the corresponding sequence is
\[y_k = \left\{10(1/2)^k -10(-1/2)^k -16\epsilon_k\right\}.\]


%----------------------------------------------------------------
% The end of notes
% ----------------------------------------------------------------
\endinput

% Local Variables:
% TeX-master: "lecture10"
% End:
