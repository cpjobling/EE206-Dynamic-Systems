%% State Space Modelling of Dynamic Systems
%% Lecture 2: Transforms of State-Space Models
\def\FileDate{99/10/06}
\def\FileVersion{1.2}
% ----------------------------------------------------------------
% Notes pages *********************************************************
% ----------------------------------------------------------------

The Laplace transform can be used to convert a differential equation
into a transfer function. It can also be used to convert a state space
model into a transfer function. In this lecture we demonstrate how
this is done and we give an example.

\ifslidesonly
\begin{slide}
  \heading{Laplace transform of a vector of functions}
  The Laplace transform of a vector $\mathbf{v}(t)$ is a vector
$\mathbf{V}(s)$. The elements of $\mathbf{V}(s)$ are the Laplace
transforms of the corresponding elements of the vector
$\mathbf{v}(t)$.
For array
\begin{displaymath}
  \mathbf{v}(t) = \left[
  \begin{array}{c}
    v_1(t) \\
    v_2(t) \\
    \vdots \\
    v_n(t) \\
  \end{array}
  \right]
\end{displaymath}


\endinput

%%% Local Variables: 
%%% mode: latex
%%% TeX-master: "notes"
%%% End: 

\end{slide}
\fi
The Laplace transform of a vector $\mathbf{v}(t)$ is a vector
$\mathbf{V}(s)$. The elements of $\mathbf{V}(s)$ are the Laplace
transforms of the corresponding elements of the vector
$\mathbf{v}(t)$.
For array
\begin{displaymath}
  \mathbf{v}(t) = \left[
  \begin{array}{c}
    v_1(t) \\
    v_2(t) \\
    \vdots \\
    v_n(t) \\
  \end{array}
  \right]
\end{displaymath}


\endinput

%%% Local Variables: 
%%% mode: latex
%%% TeX-master: "notes"
%%% End: 


\ifslidesonly
\begin{slide}
  \heading{Laplace transform of a vector of functions (cont)}
  The transformed variables are
\begin{displaymath}
  \mathcal{L}\mathbf{v}(t) = \left[
  \begin{array}{c}
   \mathcal{L} v_1(t) \\
   \mathcal{L} v_2(t) \\
    \vdots \\
   \mathcal{L} v_n(t) \\
  \end{array}
  \right] = \left[
  \begin{array}{c}
   V_1(s) \\
   V_2(s) \\
    \vdots \\
   V_n(s) \\
  \end{array}
  \right] = \mathbf{V}(s)
\end{displaymath}

\endinput
%%% Local Variables: 
%%% mode: latex
%%% TeX-master: "notes"
%%% End: 

\end{slide}
\fi
The transformed variables are
\begin{displaymath}
  \mathcal{L}\mathbf{v}(t) = \left[
  \begin{array}{c}
   \mathcal{L} v_1(t) \\
   \mathcal{L} v_2(t) \\
    \vdots \\
   \mathcal{L} v_n(t) \\
  \end{array}
  \right] = \left[
  \begin{array}{c}
   V_1(s) \\
   V_2(s) \\
    \vdots \\
   V_n(s) \\
  \end{array}
  \right] = \mathbf{V}(s)
\end{displaymath}

\endinput
%%% Local Variables: 
%%% mode: latex
%%% TeX-master: "notes"
%%% End: 


\ifslidesonly
\begin{slide}
  \heading{Example}
  \begin{eqnarray*}
  \frac{dv_{31}}{dt} &=& \frac{1}{C}\ i_2 = -\frac{1}{C}i_1 +
  \frac{1}{C} u\\
  \frac{di_{i}}{dt} &=& \frac{1}{L}\ v_{32} = \frac{1}{L}v_{31} -\frac{R}{L} i_1
\end{eqnarray*}
\endinput
%%% Local Variables: 
%%% mode: latex
%%% TeX-master: "notes"
%%% End: 

\end{slide}
\fi
\begin{eqnarray*}
  \frac{dv_{31}}{dt} &=& \frac{1}{C}\ i_2 = -\frac{1}{C}i_1 +
  \frac{1}{C} u\\
  \frac{di_{i}}{dt} &=& \frac{1}{L}\ v_{32} = \frac{1}{L}v_{31} -\frac{R}{L} i_1
\end{eqnarray*}
\endinput
%%% Local Variables: 
%%% mode: latex
%%% TeX-master: "notes"
%%% End: 

\ifslidesonly
  \begin{slide}
    \heading{Transform of state equations}
      \left[\begin{array}{c}
    dv_{31}/dt \\
    di_{i}/dt \\
    \end{array}\right]=\left[\begin{array}{cc}
      0 & -1/C \\
      1/L & -R/L \\
    \end{array}\right]\left[\begin{array}{c}
      v_{31} \\
      i_1 \\
    \end{array}\right]+\left[\begin{array}{c}
      1/C \\
      0 \\
    \end{array}\right]\ u
\endinput
%%% Local Variables: 
%%% mode: latex
%%% TeX-master: "notes"
%%% End: 

  \end{slide}
\fi
  \left[\begin{array}{c}
    dv_{31}/dt \\
    di_{i}/dt \\
    \end{array}\right]=\left[\begin{array}{cc}
      0 & -1/C \\
      1/L & -R/L \\
    \end{array}\right]\left[\begin{array}{c}
      v_{31} \\
      i_1 \\
    \end{array}\right]+\left[\begin{array}{c}
      1/C \\
      0 \\
    \end{array}\right]\ u
\endinput
%%% Local Variables: 
%%% mode: latex
%%% TeX-master: "notes"
%%% End: 


For the system in the example the state vector is defined as
$\mathbf{x}=[v_{31}, i_1]^{T}$, the input current is $u$, and the output
variables are all the currents and voltages in the circuit
$\mathbf{y}=[v_{31}, i_1, v_{32}, v_{21}, i_2]^{T}$. The transformed
state space model is therefore that shown in
\sref{slide:slide-l2eg1}.
\begin{slide}\label{slide:slide-l2eg1}
\heading{Transformed State-Equations for the Example}
\begin{eqnarray*} s\left[\begin{array}{c}
  V_{31} \\
  I_1
\end{array}\right]-\left[\begin{array}{c}
  v_{31}(0) \\
  i_1(0)
\end{array}\right]&=&\left[\begin{array}{cc}
  0 & -1/C \\
  1/L & -R/L
\end{array}\right]\left[\begin{array}{c}
  V_{31} \\
  I_1
\end{array}\right]+\left[\begin{array}{c}
  1/C \\
  0
\end{array}\right]\left[U\right]\\
\left[\begin{array}{c}
  V_{31} \\
  I_1 \\
  V_{32} \\
  V_{21} \\
  I_{2}
\end{array}\right]&=&\left[\begin{array}{cc}
  1 & 0 \\
  0 & 1 \\
  1 & -R \\
  0 & R \\
  0 & -1
\end{array}\right]\left[\begin{array}{c}
  V_{31} \\
  I_1
\end{array}\right]+\left[\begin{array}{c}
  0 \\
  0 \\
  0 \\
  0 \\
  1
\end{array}\right]\left[U\right].
\end{eqnarray*}
\end{slide}

\ifslidesonly
\begin{slide}
\heading{Transfer function from state space models (1)}
The vector $[v_{31}, i_{1}]^T$ is called the ``\emph{state
vector}.'' Its elements are state variables.
\endinput
%%% Local Variables: 
%%% mode: latex
%%% TeX-master: "notes"
%%% End: 

\end{slide}\fiThe vector $[v_{31}, i_{1}]^T$ is called the ``\emph{state
vector}.'' Its elements are state variables.
\endinput
%%% Local Variables: 
%%% mode: latex
%%% TeX-master: "notes"
%%% End: 

\ifslidesonly
\begin{slide}\heading{Transfer function from state space models (2)}
Substituting $\mathbf{X}$ from (\ref{eqn:def-of-x}) into
(\ref{eqn:def-of-y}) gives
\begin{equation}\label{eqn:def-of-y2}
  \mathbf{Y}=\left[\mathbf{C}\left[s\mathbf{I}-\mathbf{A}\right]^{-1}\mathbf{B}\mathbf{U}+
  \left[s\mathbf{I}-\mathbf{A}\right]^{-1}\mathbf{x}(0)\right]+\mathbf{D}\mathbf{U}\nonumber
\end{equation}
which after gathering terms and simplifying gives
\begin{equation}\label{eqn:def-of-y3}
  \mathbf{Y}=\left[\mathbf{C}\left[s\mathbf{I}-\mathbf{A}\right]^{-1}\mathbf{B}+\mathbf{D}\right]\mathbf{U}
      +\mathbf{C}
  \left[s\mathbf{I}-\mathbf{A}\right]^{-1}\mathbf{x}(0)
  \end{equation}

\endinput

%%% Local Variables: 
%%% mode: latex
%%% TeX-master: "notes"
%%% End: 

\end{slide}\fiSubstituting $\mathbf{X}$ from (\ref{eqn:def-of-x}) into
(\ref{eqn:def-of-y}) gives
\begin{equation}\label{eqn:def-of-y2}
  \mathbf{Y}=\left[\mathbf{C}\left[s\mathbf{I}-\mathbf{A}\right]^{-1}\mathbf{B}\mathbf{U}+
  \left[s\mathbf{I}-\mathbf{A}\right]^{-1}\mathbf{x}(0)\right]+\mathbf{D}\mathbf{U}\nonumber
\end{equation}
which after gathering terms and simplifying gives
\begin{equation}\label{eqn:def-of-y3}
  \mathbf{Y}=\left[\mathbf{C}\left[s\mathbf{I}-\mathbf{A}\right]^{-1}\mathbf{B}+\mathbf{D}\right]\mathbf{U}
      +\mathbf{C}
  \left[s\mathbf{I}-\mathbf{A}\right]^{-1}\mathbf{x}(0)
  \end{equation}

\endinput

%%% Local Variables: 
%%% mode: latex
%%% TeX-master: "notes"
%%% End: 

\ifslidesonly
\begin{slide}\heading{Transfer function from state space models (3)}
When the initial conditions of the state-variables are all zero,
this reduces to the transfer matrix model
\begin{equation}\label{eqn:transfer-function}
  \mathbf{Y}=\left[\mathbf{C}\left[s\mathbf{I}-\mathbf{A}\right]^{-1}\mathbf{B}+\mathbf{D}\right]\mathbf{U}
\end{equation}

\endinput

%%% Local Variables: 
%%% mode: latex
%%% TeX-master: "notes"
%%% End: 

\end{slide}\fiWhen the initial conditions of the state-variables are all zero,
this reduces to the transfer matrix model
\begin{equation}\label{eqn:transfer-function}
  \mathbf{Y}=\left[\mathbf{C}\left[s\mathbf{I}-\mathbf{A}\right]^{-1}\mathbf{B}+\mathbf{D}\right]\mathbf{U}
\end{equation}

\endinput

%%% Local Variables: 
%%% mode: latex
%%% TeX-master: "notes"
%%% End: 

\ifslidesonly
\begin{slide}\heading{Transfer function from state space models (4)}
\begin{itemize}
\item The matrix
$\mathbf{C}\left[s\mathbf{I}-\mathbf{A}\right]^{-1}\mathbf{B}+\mathbf{D}$
is the \emph{system transfer matrix}.
\item The element of the $i$-th
row and $j$-th column is the transfer function that relates the
$i$-th output transform $Y_i$ to the $j$-th input transform $U_j$.
\item
For a single-input, single-output (SISO) system, the system transfer matrix reduces to a
single element transfer function.
\end{itemize}

\endinput

%%% Local Variables: 
%%% mode: latex
%%% TeX-master: "notes"
%%% End: 
\end{slide}\fi\begin{itemize}
\item The matrix
$\mathbf{C}\left[s\mathbf{I}-\mathbf{A}\right]^{-1}\mathbf{B}+\mathbf{D}$
is the \emph{system transfer matrix}.
\item The element of the $i$-th
row and $j$-th column is the transfer function that relates the
$i$-th output transform $Y_i$ to the $j$-th input transform $U_j$.
\item
For a single-input, single-output (SISO) system, the system transfer matrix reduces to a
single element transfer function.
\end{itemize}

\endinput

%%% Local Variables: 
%%% mode: latex
%%% TeX-master: "notes"
%%% End: 

\ifslidesonly
\begin{slide}\heading{The Resolvant Matrix}
\begin{itemize}
\item The matrix $\left[s\mathbf{I}-\mathbf{A}\right]^{-1}$ is very
important. 
\item It is known as the \emph{resolvent matrix} of the system.
\item It may be written as
\begin{equation}\label{eqn:resolvent}
  \left[s\mathbf{I}-\mathbf{A}\right]^{-1} =
    \frac{\mathrm{adj}\left[s\mathbf{I}-\mathbf{A}\right]}
         {\det\left[s\mathbf{I}-\mathbf{A}\right]}.
\end{equation}
\end{itemize}

\endinput
%%% Local Variables: 
%%% mode: latex
%%% TeX-master: "notes"
%%% End: 
\end{slide}\fi\begin{itemize}
\item The matrix $\left[s\mathbf{I}-\mathbf{A}\right]^{-1}$ is very
important. 
\item It is known as the \emph{resolvent matrix} of the system.
\item It may be written as
\begin{equation}\label{eqn:resolvent}
  \left[s\mathbf{I}-\mathbf{A}\right]^{-1} =
    \frac{\mathrm{adj}\left[s\mathbf{I}-\mathbf{A}\right]}
         {\det\left[s\mathbf{I}-\mathbf{A}\right]}.
\end{equation}
\end{itemize}

\endinput
%%% Local Variables: 
%%% mode: latex
%%% TeX-master: "notes"
%%% End: 


For the system in the example, the resolvent matrix is developed
in \sref{slide:l2eg2}.
\begin{slide}\label{slide:l2eg2}
\heading{Resolvent matrix for the example}
\begin{eqnarray*}
\mathbf{A}&=&\left[\begin{array}{cc}
  0 & -1/C \\
  1/L & -R/L
\end{array}\right] \\
s\mathbf{I}-\mathbf{A}&=&\left[
\begin{array}{cc}
  s & 0 \\
  0 & s
\end{array}
\right] - \left[
\begin{array}{cc}
  0 & -1/C \\
  1/L & -R/L
\end{array}
\right] = \left[
\begin{array}{cc}
  s & +1/C \\
  -1/L & s+R/L
\end{array}\right] \\
\left[s\mathbf{I}-\mathbf{A}\right]^{-1}&=&\frac{
\left[\begin{array}{cc}
  s+R/L & -1/C \\
  +1/L & s
\end{array} \right]
}{s(s+R/L)+1/(LC)}=\frac{\left[\begin{array}{cc}
  s+R/L & -1/C \\
  +1/L & s
\end{array} \right]}{s^2 + (R/L)s + 1/(LC) }
\end{eqnarray*}
\end{slide}

When $\left[s\mathbf{I}-\mathbf{A}\right]^{-1}$ has been obtained,
then the system transfer function is easily obtained through
$\mathbf{C}\left[s\mathbf{I}-\mathbf{A}\right]^{-1}\mathbf{B}+\mathbf{D}$.
For the system in the example, when all outputs are measured, the
system transfer matrix is that shown in \sref{slide:l2eg3}.

\begin{slide}\label{slide:l2eg3}
\heading{Transfer matrix for example}
\begin{eqnarray*}
\left[\begin{array}{cc}
  1 & 0 \\
  0 & 1 \\
  1 & -R \\
  0 & R \\
  0 & -1
\end{array}\right]
\left\{\frac{\left[
\begin{array}{cc}
  s+R/L & -1/C \\
 +1/L & s
\end{array}\right]}{s^2 + (R/L)s +1/(LC)}\right\}
\left[\begin{array}{c}
  1/C \\
  0
\end{array}\right]
 &+& \left[\begin{array}{c}
  0 \\
  0 \\
  0 \\
  0 \\
  1
\end{array} \right]\\
= \left[\begin{array}{cc}
  1 & 0 \\
  0 & 1 \\
  1 & -R \\
  0 & R \\
  0 & -1
\end{array}\right]
\left\{\frac{\left[
\begin{array}{c}
  (1/C)s+R/(LC)  \\
 +1/(LC)
\end{array}\right]}{s^2 + (R/L)s +1/(LC)}\right\}
 &+& \left[\begin{array}{c}
  0 \\
  0 \\
  0 \\
  0 \\
  1
\end{array} \right]\\
\end{eqnarray*}
\end{slide}
\begin{slide}
\heading{Transfer matrix for example (completed)}
\begin{eqnarray*} &=& \frac{1}{s^2 + (R/L)s
+1/(LC)}\left[\begin{array}{c}
  (1/C)s+R/(LC) \\
  1/(LC) \\
  (1/C)s \\
  R/(LC) \\
  -1/(LC)
\end{array}\right] + \left[\begin{array}{c}
  0 \\
  0 \\
  0 \\
  0 \\
  1
\end{array} \right]\\
&=& \left[\begin{array}{c}
  \frac{(1/C)s+R/(LC)}{s^2 + (R/L)s
+1/(LC)} \\
  \frac{1/(LC)}{s^2 + (R/L)s
+1/(LC)} \\
  \frac{(1/C)s}{s^2 + (R/L)s
+1/(LC)} \\
  \frac{R/(LC)}{s^2 + (R/L)s
+1/(LC)} \\
  -\frac{1/(LC)}{s^2 + (R/L)s
+1/(LC)}+1
\end{array}\right]
\end{eqnarray*}
\end{slide}

In matrix form, when combined with the input and output transforms we
have the situtaion illustrated in \sref{slide:l2final}.
Each transfer function relates the corresponding output transform
to the input transform. For example
\[V_{31}= \frac{(1/C)s+R/(LC)}{s^2 + (R/L)s
+1/(LC)}\ U.\]

\begin{slide}
  \label{slide:l2final}
  \heading{Transform Equations for Example}
\begin{eqnarray*}
\mathbf{Y}(s)&=&\left[\mathbf{C}\left[s\mathbf{I}-\mathbf{A}\right]^{-1}\mathbf{B}+\mathbf{D}\right]\mathbf{U}(s)\\
\mathbf{Y}(s) &=& \mathbf{G}(s)\mathbf{U}(s)\\
\left[\begin{array}{c}
  V_{31}(s) \\
  I_1(s) \\
  V_{32}(s) \\
  V_{21}(s) \\
  I_{2}(s)
\end{array}\right] &=& \left[\begin{array}{c}
  \frac{(1/C)s+R/(LC)}{s^2 + (R/L)s
+1/(LC)} \\
  \frac{1/(LC)}{s^2 + (R/L)s
+1/(LC)} \\
  \frac{(1/C)s}{s^2 + (R/L)s
+1/(LC)} \\
  \frac{R/(LC)}{s^2 + (R/L)s
+1/(LC)} \\
  -\frac{1/(LC)}{s^2 + (R/L)s
+1/(LC)}+1
\end{array}\right]\mathbf{U}(s).
\end{eqnarray*}
\end{slide}
Note that the denominator is the same for each transfer function,
and that the order of the numerator is less than the denominator
except for one case, for which
\begin{eqnarray*}
I_2 & = & \left(-\frac{1/(LC)}{s^2 + (R/L)s +1/(LC)}+1\right)U\\
    & = & \frac{-1/{LC}+s^2 + (R/L)s +1/(LC)}{s^2 + (R/L)s
    +1/(LC)}\ U
    \\
    & = &\frac{s^2 + (R/L)s}{s^2 + (R/L)s
    +1/(LC)}\ U
\end{eqnarray*}
Replacing $s$ by $\frac{d}{dt}$ gives the corresponding
differential equations relating the dependant variable to the
input.


%----------------------------------------------------------------
% The end of slides
% ----------------------------------------------------------------
\endinput

%%% Local Variables: 
%%% mode: latex
%%% TeX-master: "#"
%%% End: 
