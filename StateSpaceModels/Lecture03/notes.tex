%% State Space Modelling of Dynamic Systems
%% Lecture 4: Time Response of a State-Space Model
\def\FileDate{98/10/14}
\def\FileVersion{1.0}
% ----------------------------------------------------------------
% Notes pages *********************************************************
% ----------------------------------------------------------------

In this lecture we shall determine the time response of a system
represented by a state-space model. We shall take as a starting
point the transformed state space model developed in the last
lecture. It is possible to derive the time response directly in
the time domain from the state-equations. I believe however, that
the following development will be easier to understand. I advise
you to consult any text book for the time domain development.

In the last lecture, we showed that the state space model was
\ssmodeltd{} and that the transformed equation was  \ssmodelfd{}
In this lecture we will show how the time response of the state
space model may be derived from the transformed model.

\section*{Zero Input Response}
\label{sec:zero_input_response} We first consider the response of
the system to its initial conditions and zero input. This is the
\emph{zero input response} or homogeneous response of the system.
With the input transform $U(s)=0$ the state equation becomes
\[\mathbf{X}(s)=\resolvent{}\mathbf{x}(0).\] We define
\[\Phis{} = \resolvent{}\] so that
\begin{equation}\label{eq:l4eq1}
\mathbf{X}(s)=\Phis{}\mathbf{x}(0)
\end{equation} is the
transform of the zero input state response. In the time domain,
the corresponding state response is given by the inverse Laplace
transform of equation~(\ref{eq:l4eq1})
\begin{equation}
\label{eq:l4eq2} \mathbf{x}(t) = \mathcal{L}^{-1}
\left\{\Phis{}\mathbf{x}(0)\right\},\end{equation} which, since
the initial condition vector $\mathbf{x}(0)$ is a vector of
constants becomes:
\begin{eqnarray}
\mathbf{x}(t) & = & \mathcal{L}^{-1}
\left\{\Phis{}\right\}\mathbf{x}(0)\nonumber \\ & = &
\phit{}\mathbf{x}(0).\label{eq:l4eq3}
\end{eqnarray}
\section*{State Transition
Matrix}\label{sec:state_transition_matrix} The vector of time
functions $\phit{}$ is known as the \emph{state transition
matrix}. It defines the motion (or \emph{trajectory}) of the state
variables through the $n$-dimensional \emph{state space} from a
given set of initial states.
\begin{slide}\label{slide:l4slide1}
\heading{State Transition Matrix} The \emph{resolvent matrix} is
defined as \[\Phis{} = \resolvent =
\frac{\adjsIminusA}{\detsIminusA}.\] The resolvent matrix is a
matrix of rational polynomials in the Laplace transform variable
$s$. The state transition matrix is
\[\phit{}
=
\mathcal{L}^{-1}\left\{\frac{\adjsIminusA}{\detsIminusA}\right\}.\]
What is the form of this function?
\end{slide}
\begin{slide}\label{slide:l4slide2}
\heading{Characteristic Polynomial} The adjoint matrix
$\adjsIminusA$ is a matrix of polynomials each of which is of
order $n-1$ or less. The determinant
\[\detsIminusA{} = \left|\begin{array}{cccc}
  s-a_{11} & -a_{12} & \cdots & -a_{1n} \\
  -a_{21} & s-a_{22} & \cdots & -a_{2n} \\
  \vdots & \vdots & \ddots & \vdots \\
  -a_{n1} & -a_{2n} & \cdots & s-a_{nn}
\end{array}\right|\] is called the \emph{characteristic polynomial}. It
is always a polynomial of degree $n$. Therefore each element of
the resolvent matrix $\Phis{}$ is of the form
\[\frac{b_{n-1}s^{n-1}+ \cdots + b_0}{s^n + a_{n-1}s^{n-1}+\cdots+a_0}.\]
\end{slide}
\begin{slide}\label{slide:l4slide3}
\heading{System Characteristic (or Eigen) Values} The coefficients
$b_i$ of each element of the resolvent matrix $\Phis{}$ depend on
the form of the system. The characteristic polynomial
$\detsIminusA$ is the common denominator of all elements of
$\Phis{}$. It may be factorized:
\begin{eqnarray*}
\detsIminusA{}&=&s^n + a_{n-1}s^{n-1} + \cdots + a_1s + a_0
\\
&=&(s - p_1)(s-p_2)\cdots(s-p_n)
\end{eqnarray*}
The values $p_i$ are known as the \emph{characteristic} (or
\emph{eigen}) values of the state matrix $\mathbf{A}$. The
characteristic values are the \emph{poles} of the
system!
\end{slide}
\begin{slide}\label{slide:l4slide4}
\heading{The Zero Input Response} If we make a partial fraction
expansion of the characteristic equation, then for the $i$-$j$th
element of the resolvent matrix we have
\[\Phi_{ij}(s)=\frac{r_1}{s-p_1}+\frac{r_2}{s-p_2} + \cdots +
\frac{r_n}{s-p_n}\](where the coefficient $r_i$ is the so called
\emph{residue} of the partial fraction expansion determined at the
value $s=p_i$). Inverse Laplace transforming this equation we
finally obtain the \emph{state transition} function:
\[\phi_{ij}(t) = r_1 e^{p_1 t} + r_2 e^{p_2 t} + \cdots + r_n
e^{p_n t}.\]
\end{slide}
\begin{slide}\label{slide:l4slide5}
\heading{System ``Modes''} The state transition matrix $\phit{}$
is an $n\times n$ matrix, each element of which is a linear
combination of the $n$ ``mode functions'' (or simply \emph{modes})
\[e^{p_1 t},\ e^{p_2 t},\ \ldots,\ e^{p_n t}.\] Each mode is constructed from the
eigen values ($p_1,\ p_2,\ \ldots,\ p_n$) of the state matrix
$\mathbf{A}$. The eigen values are also known as the \emph{poles}
of the system.
\end{slide}
\subsection*{Example}
If \[\mathbf{A} = \left[\begin{array}{cc}
  0 & 1 \\
  -6 & -5
\end{array}\right]\] \[\sIminusA = \left[\begin{array}{cc}
  s & -1 \\
  6 & s+5
\end{array}\right]\] so \[\detsIminusA = s^2 + 5s + 6 = (s+2)(s+3).\]
The characteristic values of the system are $p_1=-2$ and $p_2=-3$
so the modes are $e^{-2t}$ and $e^{-3t}$.

Now \[\adjsIminusA = \left[\begin{array}{cc}
  s+5 & 1 \\
  -6 & s
\end{array}\right]\] so the resolvent matrix is \[\Phis{} = \left[\begin{array}{cc}
  \frac{s+5}{(s+2)(s+3)} & \frac{1}{(s+2)(s+3)} \\
  \frac{-6}{(s+2)(s+3)} & \frac{s}{(s+2)(s+3)}
\end{array}\right].\] Expanding each element of $\Phis$ we get
\[\Phis{} = \left[\begin{array}{cc}
  \frac{3}{s+2}-\frac{2}{s+3} & \frac{1}{s+2}-\frac{1}{s+3} \\
  \frac{-6}{s+2}+\frac{6}{s+3} & -\frac{2}{s+2}+\frac{3}{s+3}
\end{array}\right].\] Inverse Laplace transforming this we get the final state
transition matrix, representing the zero-input response of the
system: \[\phit{}=\left[\begin{array}{cc}
  3e^{-2t}-2e^{-3t} & e^{-2t}-e^{-3t} \\
 -6e^{-2t}+6e^{-3t} & -2e^{-2t}+3e^{-3t}
\end{array}\right].\]

Note that, by an alternative derivation\footnote{%
For a first order differential equation \[\frac{dx}{dt}=ax\] so
the Laplace transform is \begin{eqnarray*}sX(s) - x(0) & = &
aX(s)\\ X(s)(s - a)& = & x(0)\\ X(s)
&=&\frac{x(0)}{s-a}.\end{eqnarray*} The inverse laplace transform
of the final equation gives \[x(t) = e^{at}.\] For the vector case
\[s\mathbf{X}(s)-\mathbf{x}(0)=\mathbf{A}\mathbf{X(s)}\] so by comparison
\[\mathbf{x}(t)=e^{\mathbf{A} t}.\] The proper proof is of course
a little more involved!}, we can show that
\[\phit{}=e^{\mathbf{A}t}\] where $e^{\mathbf{A}t}$ is called the matrix exponential.

\section*{The Zero State Response}
We now consider the case where the initial condition vector
$\mathbf{x}(0)=\mathbf{0}$ and we will obtain the response of the
system to an input. This is called the \emph{zero state response}
of the system.

\begin{slide} \label{slide:l4s11}
\heading{Zero State Response}
For zero initial conditions we have
\begin{eqnarray}\label{eq:l4eq6}
  \mathbf{X}(s)&=&\resolvent{}\mathbf{B}\mathbf{U}(s) \\
&=& \Phis{}\mathbf{B}\mathbf{U}(s) \label{eq:l4eq7}
\end{eqnarray}
The system output transform (see last lecture) is given by
\begin{equation}\label{eq:l4eq8}
 \mathbf{Y}(s)=\left[\mathbf{C}\Phis{}\mathbf{B}+\mathbf{D}\right]\mathbf{U}(s)
\end{equation} (where $\mathbf{C}\Phis{}\mathbf{B}+\mathbf{D}$ is the system transfer
matrix).
\end{slide}
\begin{slide}\label{slide:l4s12}
\heading{Zero State Response (Example)} Let
\begin{eqnarray*}\dot{\mathbf{x}}&=&\left[\begin{array}{cc}
  0 & 1 \\
  6 & -5
\end{array}\right]\mathbf{x}+\left[\begin{array}{cc}
  2 & 1 \\
 -1 & 0
\end{array}\right]\mathbf{u}\\
\mathbf{y}&=&\left[\begin{array}{cc}
  2 & -1 \\
  0 & 1
\end{array}\right]\mathbf{x}\\ \mathbf{u}&=&\left[\begin{array}{c}
  e^{-t} \\
  \epsilon(t)
\end{array}\right].
\end{eqnarray*}
\end{slide}
\begin{slide}\label{slide:l4s13}
\heading{Zero State Response: Transfer Function} The transfer
function is \begin{eqnarray*}
\mathbf{G}(s)&=&\mathbf{C}\Phis{}\mathbf{B} =
\left[\begin{array}{cc}
  2 & -1 \\
  0 & 1
\end{array}\right]
\left[\begin{array}{cc}
  \frac{s+5}{(s+2)(s+3)} & \frac{1}{(s+2)(s+3)} \\
  \frac{-6}{(s+2)(s+3)} & \frac{s}{(s+2)(s+3)}
\end{array}\right]\left[\begin{array}{cc}
  2 & 1 \\
 -1 & 0
\end{array}\right]\\
&=&\left[\begin{array}{cc}
  2 & -1 \\
  0 & 1
\end{array}\right]
\left[\begin{array}{cc}
  \frac{2s+9}{(s+2)(s+3)} & \frac{s+5}{(s+2)(s+3)} \\
  \frac{-s-12}{(s+2)(s+3)} & \frac{-6}{(s+2)(s+3)}
\end{array}\right]\\
&=& \left[\begin{array}{cc}
  \frac{5s+30}{(s+2)(s+3)} & \frac{2s+16}{(s+2)(s+3)} \\
   \frac{-s-12}{(s+2)(s+3)} & \frac{-6}{(s+2)(s+3)}
\end{array}\right]
\end{eqnarray*}
\end{slide}
\begin{slide}\label{slide:l4s14}
\heading{Output Transform} The transform of the input vector is
\[\mathbf{U}(s) = \mathcal{L}\left[\begin{array}{c}
  e^{-t} \\
  \epsilon(t)
\end{array}\right] = \left[\begin{array}{c}
  \frac{1}{s+1} \\
  \frac{1}{s}
\end{array}\right].\] Hence
\begin{eqnarray*}\mathbf{Y}(s) &=& \left[\begin{array}{cc}
  \frac{5s+30}{(s+2)(s+3)} & \frac{2s+16}{(s+2)(s+3)} \\
   \frac{-s-12}{(s+2)(s+3)} & \frac{-6}{(s+2)(s+3)}
\end{array}\right]\left[\begin{array}{c}
  \frac{1}{s+1} \\
  \frac{1}{s}
\end{array}\right]\\
&=&\left[\begin{array}{c}
  \frac{7s^2+48s+16}{s(s+1)(s+2)(s+3)} \\
  \frac{-(s^2+18s+6)}{s(s+1)(s+2)(s+3)}
\end{array}\right] = \left[\begin{array}{c}
  \frac{8/3}{s}+\frac{25/2}{s+1}-\frac{26}{s+2}+\frac{65/6}{s+3} \\
   -\frac{1}{s}-\frac{11/2}{s+1}+\frac{13}{s+2}-\frac{13/2}{s+3}
\end{array}\right]
\end{eqnarray*}
\end{slide}
\begin{slide}\label{slide:l4s15}
\heading{System Time Response} Inverse Laplace transforming the
previous result gives the zero-state output response of the
system: \[\left[\begin{array}{c}
  \frac{8}{3}\epsilon(t)+\frac{25}{2}e^{-t}-26e^{-2t}+\frac{65}{6}e^{-3t} \\
   -\epsilon(t)-\frac{11}{2}e^{-t}+13e^{-2t}-\frac{13}{2}e^{-3t}
\end{array}\right].\]
Note that response is now a linear combination of the system modes
($e^{-2t}$ and $e^{-3t}$) and the input modes ($\epsilon(t)$ [the
unit step function] and $e^{-t}$).
\end{slide}
\section*{Full System Response}
The full system response for the state-space model is simply the
sum of the zero-state and zero-input responses:
\begin{eqnarray*}\mathbf{Y}_{\mathrm{full}}(s) &=& \mathbf{Y}_{\mathrm{zs}}(s) +
\mathbf{Y}_{\mathrm{zi}}(s)\\ &=&
\mathbf{C}\Phis{}\left[\mathbf{x}(0)+\mathbf{B}\mathbf{U}(s)\right].\end{eqnarray*}
If the initial condition vector $\mathbf{x}(0)=[-1, 1]^T$ what is
the total response of the system?
\begin{slide}\label{slide:l4s16}
\heading{Total System Response} \begin{eqnarray*} \mathbf{Y}(s)&=&
\left[\begin{array}{cc}
  2 & -1 \\
  0 & 1
\end{array}\right]
\left[\begin{array}{cc}
  \frac{s+5}{(s+2)(s+3)} & \frac{1}{(s+2)(s+3)} \\
  \frac{-6}{(s+2)(s+3)} & \frac{s}{(s+2)(s+3)}
\end{array}\right]\\
&\times& \left\{\left[\begin{array}{c}
  -1 \\
  1
\end{array}\right]+\left[\begin{array}{cc}
  2 & 1 \\
  -1 & 0
\end{array}\right]\left[\begin{array}{c}
  \frac{1}{s+1} \\
  \frac{1}{s}
\end{array}\right]\right\}\\
&=& \left[\begin{array}{cc}
  \frac{2s+16}{(s+2)(s+3)} & \frac{-s+2}{(s+2)(s+3)} \\
  \frac{-6}{(s+2)(s+3)} & \frac{s}{(s+2)(s+3)}
\end{array}\right]\left[\begin{array}{c}
  \frac{-s^2+2s+1}{s(s+1)} \\
  \frac{s^2}{s(s+1)}
\end{array}\right]
\end{eqnarray*}
\end{slide}
\begin{slide}\label{slide:l4s17}
\heading{Total System Response (2)} \begin{eqnarray*}
\mathbf{Y}(s)&=& \left[\begin{array}{c}
  \frac{-3s^2-10s^2+34s+16}{s(s+1)(s+2)(s+3)} \\
  \frac{s^3+6s^2-12s-6}{s(s+1)(s+2)(s+3)}
\end{array}\right]\\
\mathbf{y}(t)&=& \left[\begin{array}{c}
  \frac{8}{3}\epsilon(t)+\frac{25}{2}e^{-t}-34e^{-2t}+\frac{95}{6}e^{-3t} \\
   -\epsilon(t)-\frac{11}{2}e^{-t}+17e^{-2t}-\frac{19}{2}e^{-3t}
\end{array}\right].
\end{eqnarray*}
\end{slide}
\subsection*{Problems}
\begin{enumerate}
\item Prove that the impulse response of a state-space model is
\begin{eqnarray*}
\mathbf{g}(t)&=&\mathcal{L}^{-1}\left\{\mathbf{C}\Phis{}\mathbf{B}+\mathbf{D}\right\}\\
&=&\mathcal{L}^{-1}\left\{\mathbf{G}(s)\right\}
\end{eqnarray*}
\item \label{l4q1}
For the circuit example introduced in Lecture 1, if $L=100$~mH,
$C=1000$~$\mu$F and $R=20$~$\Omega$, and the output is taken to be
the voltage $v_{21}(t)$ across the resistor, determine the
following:
\begin{enumerate}
\item The state transition matrix
\item The zero input response given an initial voltage across the
capacitor $v_{32}(0)=10$ V and assuming the current in the
inductor is initially $0$ A.
\item The zero state output and state variable responses given an input current of
$i=0.5e^{-t}$ \item The total output response
\end{enumerate}
\item Use state-space methods to find the total solution to the
differential equation
$$\frac{d^2(t)}{dt^2}+20\frac{dy(t)}{dt}+10y(t) = 16u(t)$$ given
the initial conditions $\frac{dy(0)}{dt}=1$ and $y(0)=0$ and
$u(t)=\epsilon(t)$.
\item Show that the state space models
\begin{eqnarray*}\dot{\mathbf{x}}&=&\left[\begin{array}{cc}
  0 & 1 \\
  -100 & -200
\end{array}\right]+\left[\begin{array}{c}
  0 \\
  200000
\end{array}\right]u\\ y&=&[1, 0]\mathbf{x}
\end{eqnarray*} and \begin{eqnarray*}\dot{\mathbf{x}}&=&\left[\begin{array}{cc}
  -0.50126 & 0 \\
  0 & -199.5
\end{array}\right]+\left[\begin{array}{c}
  1124.2 \\
  2.0051\times 10^{5}
\end{array}\right]u\\ y&=&[0.89398   -0.0050125]\mathbf{x}
\end{eqnarray*} have the same
input-output relationship (transfer function). Construct the state
transition matrix $\phit$ and the zero state and zero input output
response equations for both systems and observe how the modes of
the system are represented in the system output equations. Compare
the result with that of question~\ref{l4q1} which has the same
transfer function! What does this tell you about the transfer
function as a representation of a system?
\end{enumerate}

%----------------------------------------------------------------
% The end of slides
% ----------------------------------------------------------------
\endinput
